
\documentclass[pdftex,a4paper,11pt,DIV15,BCOR20mm,parskip,numbers=noenddot]{scrbook}

	\usepackage{german, ngerman} %Deutsche Trennungen, Anf�hrungsstriche und mehr
	\usepackage[ngerman,german]{babel} % dt. Sprache laden (u.a. auch Ausgabe feststehender Begriffe in deutsch wie etwa Inhaltsverzeichnis statt table of contents)
	\usepackage[latin1]{inputenc} % Zeichenkodierung laden f�r Unix/Windows-Systeme
	\usepackage[babel,german=quotes]{csquotes} % Deutsche Eingabe von �,�,�,� erlauben	
	\usepackage[T1]{fontenc}
  \usepackage[pdftex]{color} 
  \usepackage{setspace} % Das Paket setspace erm�glicht ein einfaches Umstellen von normalem, anderthalbfachen oder doppeltem Zeilenabstand.
  \usepackage{graphics} %Zum Einbinden von Grafiken
  \usepackage[pdftex]{graphicx}  
	\usepackage[final]{listings}  % erm�glicht die Darstellung von (Programm-)Quellcode in der Arbeit
  \usepackage{ae,aecompl}  % ben�tigt zur PDF-Erzeugung, vgl http://dsanta.users.ch/resources/type1.html
  \usepackage[normalem]{ulem} % erm�glicht das Unterstreichen von Text
  \usepackage{amsfonts} % ok-zeichen usw
  \usepackage{scrpage2}
  \usepackage{bibgerm}
  \usepackage{array} % f�r tabellen

	% Beliebige Farben definieren (bei Bedarf auskommentieren und selbst benennen und festlegen)
%	\definecolor{hellblau}{RGB}{238,242,247}
	
	\usepackage{colortbl} % f�r farbige Tabellen
	\usepackage{longtable} % f�r mehrseitige Tabellen
%	\usepackage{booktabs} % f�r tabellen: baut \midrule, \toprule und \bottomrule ein
%	\setlength{\tabcolsep}{30pt} % in Tabellen: Padding des Spalten nach links und rechts
	\renewcommand{\arraystretch}{1.25} % in Tabellen: Padding des Textes nach oben und unten in Prozent

	% Benutzt man bestimmte Spaltendefinitionen (hier: Spaltentyp p mit 25% Breite) �fter, kann man diese hier auslagern und unter einem bestimmten Buchstaben (hier: A) speichern
%	\newcolumntype{A}{p{0.25\textwidth}} 

	\usepackage{nomencl} % Package f�r ein Abk�rzungsverzeichnis
	\renewcommand{\nomname}{Abk�rzungsverzeichnis} % Deutsche �berschrift
	\setlength{\nomlabelwidth}{.20\hsize} % Breite eines Eintrags
	\renewcommand{\nomlabel}[1]{#1 \dotfill} % Auff�llen mit Punkten
	\setlength{\nomitemsep}{-2\parsep} % Zeilenabstand verkleinern
	\makenomenclature  % Datenbank des Abk�rzungsverzeichnisses beim Kompilieren des Textes erzeugen

	% fuer Literaturverweise
	\usepackage[numbers,square]{natbib}
	\bibpunct{[}{]}{;}{a}{}{;} % Natbib konfigurieren, so da� im text [M�ller 2006] erscheint und keine runden Klammern o.�.
	\renewcommand{\cite}{\citep} % \cite zu \citep umwandeln, so dass im Text \cite{Mueller.2006} f�r Referenzen verwendet werden kann.
%	\usepackage{url} % damit im LitVZ auch so etwas wie 'M�ller 2006: Vortrag zum Thema XY (Powerpoint-Pr�sentation)' steht; (Dateiendungen werden also textuell repr�sentiert)

	%% Verhindert, dass im Literaturverzeichnis nach der Nennung von [Mueller 2006] die Seite umgebrochen wird und die Autoren + Buchtitel etc. auf der n�chsten Seite landen
	%% Es wurde daher \noparagraphbreak in der natdin.bst bei der Ausgabe der Elemente erg�nzt
	%% siehe http://groups.google.de/group/de.comp.text.tex/browse_thread/thread/9ca4c681e2063fbd/f7eb3eec044b058e
	\makeatletter
	\def\noparagraphbreak{\interlinepenalty10000
	  \@itempenalty-\@highpenalty}
	\makeatother 


	% Fu�noten werden im gesamten Dokument fortlaufend hochgez�hlt und nicht nur kapitelweise, vgl. http://www.golatex.de/nummerierung-der-fussnoten-durchgehend-im-gesamten-dokument-t2042.html
	\usepackage{chngcntr}
	\counterwithout{footnote}{chapter}

	\setlength{\emergencystretch}{1em} % F�r den Fall, dass Zeilen im 1. Anlauf nicht richtig umgebrochen werden k�nnen, einen 'Notfallraum' einrichten (vgl. http://www.golatex.de/overfull-boxes-in-latex-t1979.html) 

	% Definitionen werden mit Hilfe von 'dfn' eingeleitet und k�nnen somit zentral formatiert werden.
	\newtheorem{dfn}{Definition}
	
	% entnommen aus http://www.siart.de/typografie/latextipps.xhtml#floats
	\renewcommand{\floatpagefraction}{0.8} % gibt den Bruchteil einer Seite, die f�r Gleitobjekte benutzt wird, an, der erreicht werden muss, bevor eine neue Seite angefangen wird. (Standard: 0.5; d.h. wenn ein Bild 51% der Seite einnimmt, wird extra f�r dieses Bild eine ganze Seite reserviert --> unsch�n)
	\renewcommand{\topfraction}      {0.8}
	\renewcommand{\bottomfraction}   {0.5} % \topfraction / \bottomfraction, gibt den Bruchteil einer Seite an, bis zu dem Gleitobjekte oben bzw. unten angeordnet werden sollen.
	\renewcommand{\textfraction}     {0.15} % gibt den Bruchteil einer Seite an, der mit Text belegt werden k�nnen muss.
	\makeatletter
	  \setlength{\@fptop}{0pt} % Wenn ein Float-Objekt allein auf einer Seite steht, soll es am oberen Rand der Seite erscheinen und nicht vertikal zentriert
	\makeatother
	
	% schrift auf palatino umstellen
  \usepackage{palatino}
  \setkomafont{sectioning}{\normalcolor\bfseries}


	% PDF-Support einbinden und konfigurieren
  \usepackage[
  	pdfstartview={Fit},   
  	pdffitwindow=true,
  	colorlinks,
  	linkcolor=black,
  	anchorcolor=black,
  	citecolor=black,
  	urlcolor=black
  ]{hyperref}
  \hypersetup
  {
  	pdftitle     = {Diplomarbeit zum Thema Sprachverarbeitung mit fokusierung auf Vocaltrennung mit hilfe eines LSTM-Netzes},
  	pdfsubject   = {Universit�t Hamburg / Department Informatik / ITG / Bachlorarbeit},
  	pdfauthor    = {Wajid Ghafoor},
  	pdfkeywords  = {Bachlorarbeit, Universit�t Hamburg}, % Hier kann eine beliebige Liste von Schl�sselw�rtern eingegeben werden, die z.B. Windows zur Indexierung/Suche benutzt
  	plainpages   = false 
  }
 
   
  % Zeilenabstand
  \setstretch{1.24}   

	% Strafpunkte, die beim Seitenumbruch vergeben werden, falls die erste Zeile eines Absatzes allein auf der vorangehenden Seite verbleibt. vgl http://www.jr-x.de/publikationen/latex/tipps/zeilenumbruch.html
	\clubpenalty=150

	% Strafpunkte, die beim Seitenumbruch vergeben werden, falls die letzte Zeile gerade noch auf die n�chste Seite umgebrochen wird. vgl http://www.jr-x.de/publikationen/latex/tipps/zeilenumbruch.html
	\widowpenalty=150  

  % Pagestyle definieren (nach Martins Template)
	\defpagestyle{diplHeadings}
	{ % es folgt: Definition des Seitenkopfes: 
	  % obere Linie
		(0pt,0pt)
		% linke Seite
		{\upshape \rlap{\pagemark} \hfill \headmark \hfill} % auf einer linken Seite soll LINKS die Seitenzahl stehen und mittig die Headline (headmark)
		% rechte Seite
		{\upshape \hfill \headmark \hfill \llap{\pagemark}} % auf einer rechten Seite soll RECHTS die Seitenzahl stehen und mittig die Headline (headmark)
		% falls Layout "one page"
		{}
		% untere Line
		(\textwidth,1pt)
	}
	{ % es folgt: Definition des Seitenfu�es: Wir wollen lediglich eine schwarze, �ber die komplette Seite gehende Linie erzeugen
	  % obere Linie
		(\textwidth,1pt)
		% linke Seite
		{}
		% rechte Seite
		{}
		% falls Layout "one page"
		{}
		% untere Linie
		(0pt,0pt)
	}  
	% Pagestyle auch f�r Chapter-Anfang einrichten
	\renewcommand*{\chapterpagestyle}{diplHeadings}
	\renewcommand*{\chapterheadstartvskip}{\vspace*{-\topskip}}
	\automark[section]{chapter}


	% R�nder	
	\setlength{\textwidth}{15cm}        % Textbreite
	\setlength{\textheight}{24cm}       % Texth�he
	\setlength{\topmargin}{-12mm}       % oberer Rand
  
 
  % Listingsformatierung (Quelltexte)
  \definecolor{lightgrey}{rgb}{0.90,0.90,0.90}
  \lstloadlanguages{XML}
  \lstset{
    tabsize=2,
    escapeinside={(*@}{@*)},
    captionpos=t,
    framerule=0pt,
    backgroundcolor=\color{lightgrey},
    basicstyle=\small\ttfamily,
    keywordstyle=\small\bfseries,
    numbers=left,
    fontadjust
  }    
%
% EOF
%